%\setcounter{secnumdepth}{4} %Nummerieren bis in die 4. Ebene
%\setcounter{tocdepth}{4} %Inhaltsverzeichnis bis zur 4. Ebene

%\pagestyle{headings}


\sloppy % LaTeX ist dann nicht so streng mit der Silbentrennung
%~ \MakeShortVerb{\§}

\parindent0mm
\parskip0.5em


{
%\textwidth170mm
%\oddsidemargin30mm
%\evensidemargin30mm
%\addtolength{\oddsidemargin}{-1in}
%\addtolength{\evensidemargin}{-1in}

%\parskip0pt plus2pt

% Die Raender muessen eventuell fuer jeden Drucker individuell eingestellt
% werden. Dazu sind die Werte fuer die Abstaende `\oben' und `\links' zu
% aendern, die von mir auf jeweils 0mm eingestellt wurden.

%\newlength{\links} \setlength{\links}{10mm}  % hier abzuaendern
%\addtolength{\oddsidemargin}{\links}
%\addtolength{\evensidemargin}{\links}

\begin{titlepage}
\vspace*{-1.5cm}
\raisebox{16mm}{
    \begin{minipage}[t]{70mm}
        \begin{center}
            %\selectlanguage{german}
            {\Large AKAD Universität\\}
            {\normalsize
                AKAD Bildungsgesellschaft mbH\\
            }
            \vspace{3mm}
            {\small Heilbronner Straße 86 -- 70191 Stuttgart\\}
        \end{center}
    \end{minipage}
}
\hfill
\raisebox{0mm}{
    \includegraphics[width=150pt]{Images/Ikon.png}}
\vspace{10em}

% Titel
\begin{center}
%    \baselineskip=55pt
    %\baselineskip=45pt
    \textbf{\huge \titel}\\
    \ifthenelse{\isempty{\langtitel}}{}{
            {\LARGE \langtitel}
    }
    \baselineskip=0 pt
\end{center}

%\vspace{7em}

%\vfill

% Autor
%\begin{center}
%    \textbf{\Large
%        \bearbeiter
%    }
%\end{center}

\begin{center}
    \textbf{\Large
        \studiengang{}
    }
\end{center}

\vspace{15mm}

% Prüfungsordnungs-Angaben
\begin{center}
%\selectlanguage{german}

%%%%%%%%%%%%%%%%%%%%%%%%%%%%%%%%%%%%%%%%%%%%%%%%%%%%%%%%%%%%%%%%%%%%%%%%%
% Ja, richtig, hier kann die BA-Vorlage zur MA-Vorlage gemacht werden...
% (nicht mehr nötig!)
%%%%%%%%%%%%%%%%%%%%%%%%%%%%%%%%%%%%%%%%%%%%%%%%%%%%%%%%%%%%%%%%%%%%%%%%%
{\Large \arbeit}


\vspace{2em}
\ifthenelse{\equal{\sprache}{deutsch}}{
    \begin{tabular}[t]{ll}
    Vorgelegt von: & \bearbeiter\\
    Matrikelnummer: & \matrikelnummer\\
    E-Mail Adresse: & \email\\
    Modul: & \akadmodule\\
    Beginn des Assignments:& \beginndatum \\
    Abgabe des Assignments:& \abgabedatum \\
    Gutachter:         & \erstgutachter \\
    & \zweitgutachter \\
}{
    \begin{tabular}[t]{ll}
    Date of issue:& \beginndatum \\
    Date of submission:& \abgabedatum \\
    Reviewers:         & \erstgutachter \\
    & \zweitgutachter \\
}
\end{tabular}
\end{center}

\end{titlepage}

}


\clearpage
\begin{titlepage}
    \vspace*{\fill}

    \section*{Erklärung}

    Ich versichere, dass ich das vorliegende \arbeit{} selbstständig verfasst, keine anderen als
    die angegebenen Quellen und Hilfsmittel benutzt sowie alle wörtlich oder sinngemäß über-
    nommenen Stellen in der Arbeit gekennzeichnet habe. Alle Teile meiner Arbeit, die wort-
    wörtlich oder dem Sinn nach anderen Werken entnommen sind, wurden unter Angabe der
    Quelle kenntlich gemacht. Gleiches gilt auch für Zeichnungen, Skizzen, bildliche Darstellun-
    gen sowie für Quellen aus dem Internet. Dazu zählen auch KI-basierte Anwendungen oder
    Werkzeuge. Die Arbeit wurde in gleicher oder ähnlicher Form noch nicht als Prüfungsleis-
    tung eingereicht. 
    
    \vspace{25 mm}

    \begin{tabular}{lc}
        \ort{}, den \abgabedatum \hspace*{2cm} & \underline{\hspace{6cm}} \\
                                                   & \bearbeiter
    \end{tabular}

    \vspace*{\fill}
\end{titlepage}

%\ifthenelse{\equal{\zweiseitig}{twoside}}
\clearpage
\linespread{1}
\pagenumbering{Roman}
\tableofcontents
%\setcounter{page}{1}
%\pagenumbering{arabic}
\setcounter{tocdepth}{2}
